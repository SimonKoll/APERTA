\section{Projektverlauf}
\setauthor{David Hauser}

\section{Erkenntnisse von Benjamin Golic}
\setauthor{Benjamin Golic}

\section{Erkenntnisse von David Hauser}
\setauthor{David Hauser}

\section{Erkenntnisse von Simon Koll}
\setauthor{Simon Koll}

Das Projekt nahm immer mehr Substanz an, je weiter die Entwicklung fortschritt. Es kamen neue Ideen hinzu, die Anfangs noch gar nicht im Raum standen. Die Erkenntnisse und Inhalte des ITP-, INSY- sowie SEW-Unterrichts haben mich in der Entwicklung sehr unterstützt, da ich teilweise auf bereits vorhandenes Wissen aufbauen konnte. Gleichzeitig war es eine angemessene Aufgabe, die mich genug forderte, um die Motivation hoch zu halten. \\
Die Zusammenarbeit im Team ist eine der wichtigsten Erfahrungen, die in der Laufbahn der HTL bekommt. Es zeigt, wie wichtig die Kommunikation zwischen den Teammitgliedern ist und wie man sich gegenseitig motivieren kann.\\
Ich habe viele meiner bereits vorhandenen Kenntnisse vertiefen können, jedoch auch viel Neues über Bibliotheken wie OpenCV und Tesseract gelernt. Mit bereits bekannten und vertrauten Programmiersprachen zu arbeiten erleichterte die Entwicklung ungemein, da man bei Fehlermeldungen leichter erkennt, wo das Problem liegt und wie man es lösen kann.\\
Ich habe jedoch auch gelernt, dass ich in meiner Zukunft den Berufsweg eines Entwicklers vorraussichtlich nicht einschlagen werde, da diese Arbeit mich an manchen Punkten beinahe zur Verzweiflung brachte.
Zusammenfassend bin ich mit dem Ergebnis sowie der Leistung des Teams sehr zufrieden. 