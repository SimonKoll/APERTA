\section{Auswahl der Technologie - Client}
\setauthor{David Hauser}

\subsection{Unterschied Framework und Library}
\subsection{Technologie zur Entwicklung des Frontends}
\subsection{Angular}
\subsection{React}

\section{Auswahl der Technologie - Datenbank}
\setauthor{Benjamin Golic}

\subsection{MongoDB}
\subsection{Unterschied Relationale und nicht Relationale Datenbanken}


\section{Auswahl der Technologie - Kennzeichenerkennung[SK]}
\setauthor{Simon Koll}
Das Herz von APERTA ist die Kennzeichenerkennung. Dieses Alleinstellungsmerkmal separiert das Projekt von möglicher Konkurrenz. Um eine schnelle, und problemfreie Lösung zu liefern, versuchte das Team, eine für österreichische Kennzeichen optimierte Lösung zu implementieren. Da dies aufgrund fehlender Datensets für genannte Kennzeichen schwierig bis nicht realisier bar war, griff man auf eine API zurück, welche das aufgenommene Bild erhält und das Kennzeichen zurück liefert. Diese Abfrage wird mittels eines HTTP-Requests ausgeführt.

\section{Auswahl der Technologie - Backend[SK]}
\setauthor{Simon Koll}
\subsection{Anforderungen an das Backend}
Für das Backend kamen mehrere Technologien in Frage, wie unter anderem Java, JavaScript, Python, PHP, C$\sharp$, und viele mehr.
Um das Backend zu realisieren, muss die Technologie einige bestimmte Eigenschaften besitzen.
\begin{itemize}
    \item \textit{Java:} Die Vorteile von Java liegen in der Fehlerbehandlung, sowie in Bereichen wie Multithreading und Performanz. Die strikte Fehlerbehandlung führt dabei aber zum Verlust von Flexibilität und Kompaktheit des Codes.
    \item \textit{JavaScript:} Die Syntax von JavaScript ähnelt der von Java. Entwickelt als Scripting-Sprache für HTML, ist JavaScript einfach zu lernen und zu benutzen. Bei der Entwicklung von Websites kann JavaScript direkt in den Quellcode der HTML-Seite eingearbeitet werden. Aber auch im Backend-Bereich kann mit NodeJS in JavaScript entwickelt werden.
    \item \textit{Python:} Python ist eine der mit Abstand am leichtesten zu lesenden Programmiersprachen. Die flache Hierarchie ermöglicht ein einfaches Verständnis von Programmen und Codestücken. Weiters macht Python den Entwickler auf jeden Fehler aufmerksam, wenn dieser nicht ausdrücklich ignoriert werden soll.
    Jedoch ist Python manchmal langsamer in der Ausführung als die konkurrierenden Sprachen. Zusätzlich ist durch die Verwendung von Leerzeichen zur Einrückung ein häufiger Fehlergrund hinzugekommen.
    \item \textit{PHP:} Die PHP Syntax erinnert an eine Mischung aus C, Java und Perl. Das Ziel von PHP ist es, Entwickler schnell und einfach dynamisch generierte Webpages zu bauen. Die vermischte Syntax ist jedoch etwas chaotisch, darum ist es leicht sich in falschen Angewohnheiten zu verirren und Sicherheitslücken offen zu lassen.
  \end{itemize}
  Aufgrund vorhandener Vorkenntnisse standen für das Team 3 der oben genannten Technologien zur Auswahl:

  \begin{itemize}
    \item Java,
    \item JavaScript und
    \item Python
  \end{itemize}

  Von diesen konnte sich JavaScript durchsetzen. Die Gründe dafür waren:
  \begin{itemize}
    \item NPM,
    \item Verwendung einer NoSQL Datenbank
    \item Echtzeit-Performance
  \end{itemize}

