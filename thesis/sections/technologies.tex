\section{Auswahl der Technologie - Client}
\setauthor{David Hauser}

\subsection{Unterschied Framework und Library}
\subsection{Technologie zur Entwicklung des Frontends}
\subsection{Angular}
\subsection{React}

\section{Auswahl der Technologie - Datenbank}
\setauthor{Benjamin Golic}

\subsection{MongoDB}
\subsection{Unterschied Relationale und nicht Relationale Datenbanken}


\section{Auswahl der Technologie - Kennzeichenerkennung[SK]}
\setauthor{Simon Koll}
Das Herz von APERTA ist die Kennzeichenerkennung. Dieses Alleinstellungsmerkmal separiert das Projekt von möglicher Konkurrenz. Um eine schnelle, und problemfreie Lösung zu liefern, versuchte das Team, eine für österreichische Kennzeichen optimierte Lösung zu implementieren. Da dies aufgrund fehlender Datensets für genannte Kennzeichen schwierig bis nicht realisier bar war, griff man auf eine API zurück, welche das aufgenommene Bild erhält und das Kennzeichen zurück liefert. Diese Abfrage wird mittels eines HTTP-Requests ausgeführt.

\section{Auswahl der Technologie - Backend[SK]}
\setauthor{Simon Koll}
\subsection{Anforderungen an das Backend}
Für das Backend kamen mehrere Technologien in Frage, wie unter anderem Java, JavaScript, Python, PHP, C$\sharp$, und viele mehr. 
Um das Backend zu realisieren, muss die Technologie einige bestimmte Eigenschaften besitzen. 
\begin{itemize}
    \item \textit{Java:} APERTA soll bei jedem Garagentor mit bereits verbautem Motor nachrüstbar sein. So können mehr potentielle Käufer angesprochen werden.
    \item \textit{JavaScript:} Aufgrund der vielen Möglichkeiten kann APERTA für manche Käufer in der Vollausstattung nicht geeignet sein. Das Produkt soll daher im Onlineshop konfiguriert werden können, sodass bestimmte Komponenten entfernt werden können. Diese sollen nachträglich eingebaut werden können.
    \item \textit{Python:} Um den Nutzer zu unterstützen, soll die Verwaltung von Nummernfeldkombinationen, NFC-Details und Kennzeichen einfach und schnell funktionieren. Der Käufer soll mithilfe von Texteingaben neue Einträge hinzufügen können und diese mit einem Knopfdruck wieder entfernen können.
    \item \textit{PHP:}
    Um die Erfahrung für den Käufer von Anfang an gut zu gestalten, soll ein optisch ansprechender Onlineshop der erste Kontakt mit dem Produkt sein.
    \item \textit{C$\sharp$: }
  \end{itemize}

