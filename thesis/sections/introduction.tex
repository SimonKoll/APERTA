\section{Problemsituation [SK]}
Autos sind aus dieser Welt nicht mehr wegzudenken. Alleine in Österreich sind mehr als 5 Millionen Personenkraftwagen zugelassen. Dieser Bestand wuchs nach Angaben der Statistik Austria seit mehr als 15 Jahren kontinuierlich an.\cite{StatAustPKW} Um die Sicherheit der Insassen gewährleisten zu können, wird empfohlen, das Fahrzeug in einer Garage oder einem überdachten Gebiet abzustellen. Dort sind Umwelteinflüsse wie Hagel keine große Gefahr mehr. Neue Garagen haben oftmals elektrische Tore, die mit einem Nummernfeld oder einem Handsender aus dem Auto selbst geöffnet werden können. Diese Handsender können jedoch bei der Anfahrt an die Garage in der Eile nicht gefunden werden, oder die Batterie kann sich unbemerkt entleeren. Falls zudem kein Nummernfeld oder ähnliches vorhanden ist, gibt es bei Verlust des Handsenders keine Möglichkeit, die Garage zu öffnen. Somit muss aus dem Auto ausgestiegen werden und die Notentriegelung des Systems betätigt werden.
Mit der immer weiter voranschreitenden Vernetzung der nahen Umwelt, wie Haustüren mit Fingerabruck-Zugangskontrolle oder mit dem Smartphone bedienbaren Jalousien, kommt mit APERTA das IOT in die Garage.
APERTA ist ein Komplettsystem, welches bei Garagen mit elektrischem Tor nachgerüstet werden kann. Es besteht die Möglichkeit, mithilfe eines Nummernfelde,s oder einer NFC-Karte das Garagentor wie gewohnt zu öffnen. Was APERTA aber auszeichnet ist eine integrierte Kennzeichenerkennung, bei der kein Handsender oder ähnliches mehr benötigt wird. Das Kennzeichen wird über das Web-Dashboard eingegeben, und das Tor kann dann bei Annäherung an das Tor dieses erkennen und steuert den Toröffnungsmechanismus an. Dazu wird ein Relais verwendet, welches wie handelsübliche Handschalter an der Innenseite der Garage den Steuerstromkreis der Garage schließt.

\section{Ziele[SK]}
Das Team hat sich vor Entwicklungsbeginn einige Ziele gesteckt, welche das Projekt erfüllen muss, um einen Mehrwert für potentielle Kunden zu bieten. Zu diesen Zielen zählen:
\begin{itemize}
    \item \textit{Herstellerunabhängigkeit:} APERTA soll bei jedem Garagentor mit bereits verbautem Motor nachrüstbar sein. So können mehr potentielle Käufer angesprochen werden.
    \item \textit{Modularität:} Aufgrund der vielen Möglichkeiten kann APERTA für manche Käufer in der Vollausstattung nicht geeignet sein. Das Produkt soll daher im Onlineshop konfiguriert werden können, sodass bestimmte Komponenten entfernt werden können. Diese sollen nachträglich eingebaut werden können.
    \item \textit{Übersichtliche Verwaltung:} Um den Nutzer zu unterstützen, soll die Verwaltung von Nummernfeldkombinationen, NFC-Details und Kennzeichen einfach und schnell funktionieren. Der Käufer soll mithilfe von Texteingaben neue Einträge hinzufügen können und diese mit einem Knopfdruck wieder entfernen können.
    \item \textit{Intuitiver Bestellvorgang:}
    Um die Erfahrung für den Käufer von Anfang an gut zu gestalten, soll ein optisch ansprechender Onlineshop der erste Kontakt mit dem Produkt sein.
  \end{itemize}