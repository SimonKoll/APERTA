Das der Raspberry Pi holt sich die gesamten Nummernfeldkombinationen von der Datenbank. Diese werden dann mit dem Eingegebenen Code abgeglichen und auf Übereinstimmungen geprüft. Stimmt der eingegebene Code und die Kombination der Datenbank überein, so öffnet sich das Garagentor. 
\begin{lstlisting}[language=python, caption=checkNumpad.py]
    import requests
    import json
    from relais import initiateOpeningSequence
    from testDisplay import displayText, readNumpadUnsuccessful
    
    def checkNumpad(numpad_code):
        inList = False
        r = requests.get('http://130.162.215.116/get-numpad-codes')
        val = json.loads(r.text)
        for value in val:
            print("value: ", value["numpad_code"])
            print("text from reader: ", numpad_code)
            if value["active"]:
                if numpad_code == value["numpad_code"]:
                    print("numpad code recognized, initiating opening sequence")
                    inList = True
                else:
                    print("not recognized, staying closed")
                    readNumpadUnsuccessful()
        if inList:
            displayText(str(numpad_code))
            initiateOpeningSequence()    
\end{lstlisting}

Die von der Datenbank erhaltenen Codes werden in einer Schleife durchlaufen. 

Das Keypad muss mit dem Raspberry Pi über 8-pins verbunden werden. Die Pins für die Spalten sind Outputs und werden auf high gesetzt. Die Reihenpins dienen als Input, sie sind standardmässig auf high gesetzt. Der Output wird einmal auf low gesetzt, danach werden alle Inputs in kürzester Zeit abgefragt. Sollte es einen Wert geben, welcher auf low gesetzt ist, kann genau gesagt werden welche Taste gedrückt wurde. 
\begin{lstlisting}[language=python, caption=Numpad.py]
    import RPi.GPIO as GPIO
import time
from testDisplay import displayText
from relais import initiateOpeningSequence
from testDisplay import displayText, readNumpadUnsuccessful
import requests
import json

def numpad():
    GPIO.setmode(GPIO.BCM)

    MATRIX = [[1, 2, 3, 'A'],
    [4, 5, 6, 'B'],
    [7, 8, 9, 'C'],
    ['*', 0, '#', 'D']]


    COUNT = 0
    COMBIN = ""
    ROW = [18, 23, 24, 26]
    COL = [12, 16, 20, 21]

    for j in range(4):
        GPIO.setup(COL[j], GPIO.OUT)
        GPIO.output(COL[j], 1)

    for i in range(4):
        GPIO.setup(ROW[i], GPIO.IN, pull_up_down=GPIO.PUD_UP)

    try:
        while True:
            for j in range(4):
                GPIO.output(COL[j], 0)
                for i in range(4):
                    if GPIO.input(ROW[i]) == 0:
                        print(MATRIX[i][j])
                        COUNT = COUNT+1
                        COMBIN = COMBIN + str(MATRIX[i][j])
                        if(COUNT > 5):
                            #displayText(COMBIN)
                            #checkNumpad(COMBIN)
                            r = requests.get('http://130.162.215.116/get-numpad-codes')
                            val = json.loads(r.text)
                            for value in val:
                                print("value: ", value["numpad_code"])
                                print("text from reader: ", COMBIN)
                                if value["active"]:
                                    if COMBIN == value["numpad_code"]:
                                        print("numpad code recognized, initiating opening sequence")
                                        displayText(COMBIN)
                                        initiateOpeningSequence()
                                    else:
                                        print("not recognized, staying closed")
                                        readNumpadUnsuccessful()
                            COMBIN = ""
                            COUNT = 0
                        while(GPIO.input(ROW[i]) == 0):
                            pass
                            time.sleep(0.2)
                GPIO.output(COL[j], 1)
    finally:
        #GPIO.cleanup()
        print("end of numpad")

\end{lstlisting}

Mittels einer For-Schleife werden die Spaltenpins auf high gesetzt, die Pins in der Reihe werden mit einem pull-up Resistor versehen. Es läuft eine Endlosschleife, diese wartet darauf das ein Feld gedrückt wird.