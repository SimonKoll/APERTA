Das Display kann mithilfe von drei verschiedenen Methoden unterschiedliche Texte anzeigen. 
\begin{enumerate}
    \item Kombination des Nummernfeldes korrekt
    \begin{compactenum}
        \item Bei korrekter Eingabe des Nummernfeldes wird das Display wie folgt beschrieben:
        \begin{lstlisting}[language=Python, caption=Ausgabe bei korrekter Eingabe der Kombination auf dem Nummernfeld, label=code_display_correct]
        outString = writingString
        lcd.text("Combination:", 1)
        lcd.text(outString, 2)
        \end{lstlisting}
        Nach einer Wartezeit von 5 Sekunden wird das Display geleert und steht für neue Aufgaben zur Verfügung.
    \end{compactenum}
    \newpage
    \item Kombination des Nummernfeldes nicht korrekt
    \begin{compactenum}
        \item Bei inkorrekter Eingabe des Nummernfeldes soll nicht ausgegeben werden, an welcher Stelle der Eingabe sich der Fehler befindet. Daher gibt das Display \verb|Wrong Combination, please try again| aus.
    \end{compactenum}
    \item NFC-Tag nicht korrekt
    \begin{compactenum}
        \item Bei nicht authorisierten NFC-Karten sowie bei fehlerhaften Lesevorgängen muss das Garagentor verschlossen bleiben. Daher wird in diesem Fall vom Display \verb|Error at reading, please try again| ausgegeben.
    \end{compactenum}
\end{enumerate}