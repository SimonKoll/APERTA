\begin{lstlisting}[language=Python,caption=checkLicensePlate code,label=lst:impl:checkLicensePLate]
    import requests
    import json
    import string
    from relais import initiateOpeningSequence
    
    
    def checkPlate(text):
        print(type(text))
        r = requests.get('http://130.162.215.116/get-licenseplates')
        val = json.loads(r.text)
        table = str.maketrans('', '', string.ascii_lowercase)
        text = text.strip()
        for value in val:
            print("value: ", value["licenseplate"])
            print("value no whitespace:", value["licenseplate"].replace(" ", ""))
            print("text from anpr: ", text)
            if value["active"]:
                print(text.replace(" ", "").translate(table) ==
                      value["licenseplate"].translate(table).replace(" ", ""))
                if text.replace(" ", "").translate(table) == value["licenseplate"].translate(table).replace(" ", ""):
                    print("licenseplate recognized, initiating opening sequence")
                    initiateOpeningSequence()
                else:
                    print("not recognized, staying closed")
    
    \end{lstlisting}

    \begin{lstlisting}[language=Python, caption=ANPR code, label=lst:impl:ANPR]
import cv2, os
import imutils
import numpy as np
import pytesseract
import re
from checkLicensePlate import checkPlate
def npr():
    cap = cv2.VideoCapture(0)
    if not cap.isOpened():
        print("cannot open camera")
        exit()
    while(True):
        ret, frame = cap.read()
        if not ret:
            print("Can't receive frame (stream end?). Exiting ...")
            break
        cv2.imshow("Frame", frame)
        gray = cv2.cvtColor(frame, cv2.COLOR_BGR2GRAY)  # convert to grey scale
        gray = cv2.bilateralFilter(gray, 11, 17, 17)  # Blur to reduce noise
        edged = cv2.Canny(gray, 30, 200)  # Perform Edge detection
        cnts = cv2.findContours(edged.copy(), cv2.RETR_TREE,
                                cv2.CHAIN_APPROX_SIMPLE)
        cnts = imutils.grab_contours(cnts)
        cnts = sorted(cnts, key=cv2.contourArea, reverse=True)[:10]
        screenCnt = None
        for c in cnts:
            peri = cv2.arcLength(c, True)
            approx = cv2.approxPolyDP(c, 0.018 * peri, True)
            if len(approx) == 4:
                screenCnt = approx
                break
        if screenCnt is None:
            detected = 0
            print("No contour detected")
            exit()
        else:
            detected = 1
        if detected == 1:
            cv2.drawContours(frame, [screenCnt], -1, (0, 255, 0), 3)
        mask = np.zeros(gray.shape, np.uint8)
        new_image = cv2.drawContours(mask, [screenCnt], 0, 255, -1,)
        new_image = cv2.bitwise_and(frame, frame, mask=mask)
        (x, y) = np.where(mask == 255)
        (topx, topy) = (np.min(x), np.min(y))
        (bottomx, bottomy) = (np.max(x), np.max(y))
        Cropped = gray[topx:bottomx+1, topy:bottomy+1]
        text = pytesseract.image_to_string(Cropped, config='--psm 11')
        print("text before replacing:", text)
        text_replaced = re.sub('[^a-zA-Z0-9 \n\.]', '', text)
        text_replaced = text_replaced.replace(" ", "")
        print("Detected Number is:", text_replaced)
        
        checkPlate(text_replaced)
        cv2.imshow("Frame", frame)
        cv2.imshow('Cropped', Cropped)
        break
    cv2.destroyAllWindows()
    cap.release()
    os.system("python new_anpr.py")
    \end{lstlisting}

    \begin{lstlisting}[language=Python, caption=Relais code, label=lst:impl:relais]
        import RPi.GPIO as GPIO
import time

in1 = 7
GPIO.setmode(GPIO.BOARD)
GPIO.setup(in1, GPIO.OUT)

GPIO.output(in1, False)

def initiateOpeningSequence():
    GPIO.output(in1,True)
    time.sleep(0.1)
    GPIO.output(in1,False)
    GPIO.cleanup()
    \end{lstlisting}